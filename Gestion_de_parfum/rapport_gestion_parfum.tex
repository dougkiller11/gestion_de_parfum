\documentclass[12pt,a4paper]{report}
\usepackage{graphicx}
\usepackage{placeins}
\usepackage[utf8]{inputenc}
\usepackage[T1]{fontenc}
\usepackage[french]{babel}
\usepackage{geometry}
\usepackage{pgfgantt}
\usepackage{fancyhdr}
\usepackage{enumitem}
\geometry{left=3cm, right=2.5cm, top=3cm, bottom=3cm}
\geometry{margin=2.5cm}

\begin{document}

%----------------------------------------------------------------------------------------
%  PAGE DE GARDE
%----------------------------------------------------------------------------------------
\begin{titlepage}
\begin{flushleft}
    \includegraphics[width=4cm]{emsi_login_logo.png}
\end{flushleft}

\vspace{1cm}
\begin{center}
    {\LARGE \textbf{École Marocaine des Sciences de l'Ingénieur}}\\[0.3cm]
    {\Large (EMSI)}\\[1.2cm]

    {\Large \textbf{4\ieme{} Année Informatique et Réseaux}}\\[1.5cm]

    \rule{\linewidth}{0.5mm}\\[0.4cm]
    {\Huge \textbf{Gestion de Parfums}}\\[0.2cm]
    {\Large \textbf{Plateforme e-commerce et back-office}}\\[0.4cm]
    \rule{\linewidth}{0.5mm}\\[1.5cm]

    {\Large Rapport de Projet}\\[2cm]

    \begin{minipage}{0.45\textwidth}
        \begin{flushleft}
            \textbf{Réalisé par :}\\
            Nawfel chorfi\\
            Youness touil
        \end{flushleft}
    \end{minipage}
    \hfill
    \begin{minipage}{0.45\textwidth}
        \begin{flushright}
            \textbf{Encadré par :}\\
            M./Mme ................................
        \end{flushright}
    \end{minipage}

    \vfill
    {\Large Année universitaire : 2024 -- 2025}
\end{center}
\end{titlepage}

%----------------------------------------------------------------------------------------
\chapter*{Avant-propos}
\addcontentsline{toc}{chapter}{Avant-propos}

Ce rapport présente la conception et la réalisation d'une application web de
\textbf{gestion de parfums} mêlant boutique en ligne et back-office de gestion.
L'objectif est de proposer une plateforme fiable, sécurisée et maintenable,
alignée sur les bonnes pratiques ASP.NET Core MVC et adaptée à un contexte
professionnel.

\vspace{0.4cm}
\noindent\textbf{Noms et prénoms :}
\begin{itemize}
    \item Nawfel chorfi
    \item Youness touil
\end{itemize}

\noindent\textbf{Projet :} Gestion de Parfums \\
\textbf{Établissement :} École Marocaine des Sciences de l'Ingénieur -- Casablanca \\
\textbf{Matière :} Développement .NET \\
\textbf{Encadrant :} ................................

%----------------------------------------------------------------------------------------
\chapter*{Remerciements}
\addcontentsline{toc}{chapter}{Remerciements}

Je remercie toutes les personnes qui ont contribué au bon déroulement de ce
projet : nos enseignants pour leurs conseils, nos proches pour leur soutien et
les membres de l'équipe pédagogique pour le cadre d'apprentissage fourni.

%----------------------------------------------------------------------------------------
\chapter*{Résumé}
\addcontentsline{toc}{chapter}{Résumé}

Le projet \textbf{Gestion de Parfums} est une application web développée en
\textbf{ASP.NET Core MVC (C\#)}. Elle combine une boutique en ligne pour la vente
de parfums et un back-office de gestion. Les principales fonctionnalités sont :
la navigation par catégories, la recherche et le filtrage, le panier d'achat,
le paiement simulé, l'historique des commandes, ainsi qu'un espace
d'administration pour gérer le catalogue, les stocks et suivre les ventes.

La sécurité repose sur l'authentification par cookies, la gestion de rôles
(\textit{Admin} / \textit{Client}), la validation côté serveur et la protection
CSRF. Les données sont stockées dans \textbf{SQL Server} via
\textbf{Entity Framework Core}. La configuration et l'initialisation (catégories
et parfums de démonstration) sont centralisées dans \texttt{Program.cs}.

%----------------------------------------------------------------------------------------
\chapter*{Introduction générale}
\addcontentsline{toc}{chapter}{Introduction générale}

La digitalisation du commerce impose des plateformes fiables et sécurisées,
capables de gérer le catalogue, les stocks et les commandes en temps réel.
Le projet \textbf{Gestion de Parfums} répond à ce besoin en proposant une
application web centrée sur l'expérience utilisateur (boutique) et la maîtrise
opérationnelle (back-office). Il s'appuie sur ASP.NET Core MVC pour la structure,
SQL Server pour la persistance et des contrôles d'accès adaptés aux profils
administrateur et client.

Les objectifs principaux sont :
\begin{itemize}
    \item Offrir une navigation claire par catégories et fiches produit détaillées.
    \item Permettre l'ajout au panier, le paiement simulé et l'historisation des commandes.
    \item Fournir aux administrateurs un tableau de bord, le CRUD du catalogue et le suivi des stocks.
    \item Garantir la sécurité applicative (authentification, rôles, CSRF, validation).
\end{itemize}

%----------------------------------------------------------------------------------------
\tableofcontents
\listoffigures
\setcounter{chapter}{0}

%========================================================================================
\chapter{Contexte général du projet}

\section{Présentation de l'environnement de travail}
Le projet est développé sous \textbf{ASP.NET Core 8 MVC} avec Razor Pages, EF Core
et SQL Server. L'interface consomme des vues Razor, Bootstrap pour le design et
utilise jQuery pour certaines interactions (formulaires, validation côté client).
La configuration est centralisée dans \texttt{appsettings.json} et
\texttt{Program.cs}, incluant la connexion SQL et l'initialisation des données
(catégories et parfums de démonstration).

\subsection{Service visé}
L'application couvre le parcours complet d'un acheteur de parfums :
\begin{itemize}
    \item Navigation et filtrage par catégories (\textit{Homme}, \textit{Femme}, \textit{Mixte}).
    \item Consultation des fiches produit (description, prix, stock, visuel).
    \item Ajout au panier, modification des quantités et choix de la taille.
    \item Paiement simulé, confirmation et récapitulatif.
    \item Historique des commandes pour le client, suivi global pour l'administrateur.
\end{itemize}

\subsection{Équipe}
Le développement est assuré par une équipe \textbf{full-stack} maîtrisant :
conception MVC, sécurisation des routes, intégration front, ORM EF Core et
déploiement (Kestrel/IIS).

\section{Contexte et problématique}
Les parfumeries qui gèrent leurs ventes avec des tableurs ou des outils hétérogènes
rencontrent des difficultés : erreurs de stock, absence de traçabilité, mises à
jour manuelles et lenteur de traitement des commandes. L'enjeu est de disposer
d'un \textbf{système intégré} combinant vitrine e-commerce et back-office fiable.

\section{Besoins et périmètre}
\textbf{Besoins fonctionnels} : navigation par catégories, recherche, panier,
paiement simulé, historique des achats, tableau de bord admin, CRUD parfums,
gestion des stocks, mise à jour des statuts de commande.

\textbf{Besoins techniques} : ASP.NET Core MVC, EF Core, SQL Server, authentification
cookies, rôles, protection CSRF, validation côté serveur et côté client, vues Razor
responsives (Bootstrap).

\textbf{Périmètre inclus} : gestion du catalogue, commandes en ligne, historique,
back-office admin, données seedées pour démarrage rapide.

\textbf{Hors périmètre} : passerelle de paiement réelle, application mobile native,
multilingue et multi-devise, intégration ERP/CRM, reporting avancé PDF/Excel.

\section{Étude et critique de l'existant}
Les solutions artisanales (Excel, inventaires papier) manquent de fiabilité et de
centralisation. Les outils obsolètes n'offrent ni API ni interface web moderne.
Les limites observées : erreurs d'inventaire, absence d'alertes, pas d'historique
exploitable et peu de sécurité sur les accès.

\section{Objectifs généraux}
\begin{itemize}
    \item Assurer un \textbf{parcours e-commerce} fluide et responsive.
    \item Maintenir une \textbf{cohérence stock / commandes} grâce au back-office.
    \item Sécuriser l'accès par rôles (\textit{Admin} / \textit{Client}) et sessions.
    \item Faciliter la maintenance via une architecture claire (MVC + services + EF Core).
\end{itemize}

\section{Contraintes techniques}
\begin{table}[h!]
\centering
\renewcommand{\arraystretch}{1.2}
\begin{tabular}{|p{4cm}|p{10cm}|}
\hline
\textbf{Catégorie} & \textbf{Description} \\
\hline
Backend & ASP.NET Core MVC, C\#, contrôleurs dédiés (\texttt{Boutique}, \texttt{Cart}, \texttt{AdminProducts}, \texttt{Account}). \\
\hline
Base de données & SQL Server, \texttt{ApplicationDbContext} (EF Core) avec migrations. \\
\hline
Sécurité & Authentification par cookies, rôles admin/client, protection CSRF, validation des entrées. \\
\hline
Front & Vues Razor, Bootstrap 5, jQuery pour interactions et validation. \\
\hline
Services & Seed initial (catégories, parfums), session panier, mapping des entités. \\
\hline
Configuration & \texttt{appsettings.json}, \texttt{Program.cs} (auth, routes, session, seeding). \\
\hline
\end{tabular}
\caption{Synthèse technologique du projet}
\end{table}

\section{Diagramme de Gantt}
\begin{figure}[h]
\centering
\ganttset{
    hgrid,
    vgrid,
    x unit=0.6cm,
    y unit chart=0.5cm,
    bar height=0.45,
    group height=0.35,
    title height=0.9,
    milestone height=0.3,
    link/.style={-latex}
}
\begin{ganttchart}[]{1}{12}
\gantttitle{Janvier 2025}{12} \\
\gantttitlelist{6,...,17}{1} \\

% Semaine 1
\ganttgroup{Semaine 1}{1}{5} \\
\ganttbar[name=task1]{Prise en main code / env}{1}{1} \\
\ganttbar[name=task2]{Revue besoins / périmètre}{1}{2} \\
\ganttbar[name=task3]{Maquettes vues boutique}{2}{3} \\
\ganttbar[name=task4]{CRUD Parfums + Catégories}{3}{5} \\

% Semaine 2
\ganttgroup{Semaine 2}{8}{12} \\
\ganttbar[name=task5]{Panier + sessions}{8}{9} \\
\ganttbar[name=task6]{Commande + paiement simulé}{9}{10} \\
\ganttbar[name=task7]{Dashboard admin (stocks, ventes)}{10}{11} \\
\ganttbar[name=task8]{Tests, démo, polissage UI}{11}{12} \\

% Dépendances
\ganttlink{task4}{task5}
\ganttlink{task5}{task6}
\ganttlink{task6}{task7}
\ganttlink{task7}{task8}

\end{ganttchart}
\caption{Diagramme de Gantt -- Gestion de Parfums (2 semaines)}
\end{figure}

%========================================================================================
\chapter{Analyse et conception}

\section{Choix UML}
UML est utilisé pour cadrer la structure et les interactions :
\begin{itemize}
    \item \textbf{Diagramme de classes} : Parfum, Categorie, Client, Commande, LigneCommande, Utilisateur.
    \item \textbf{Cas d'utilisation} : navigation boutique, ajout au panier, paiement, suivi des commandes, gestion du catalogue par l'admin.
    \item \textbf{Diagrammes de séquence} : parcours d'achat, enregistrement d'une commande et notification admin.
\end{itemize}

\section{Diagrammes}
\begin{figure}[h]
    \centering
    \includegraphics[width=12cm]{uml_classes_parfum.png} % à remplacer
    \caption{Diagramme de classes principal}
\end{figure}

\begin{figure}[h]
    \centering
    \includegraphics[width=12cm]{uml_usecase_parfum.png} % à remplacer
    \caption{Cas d'utilisation clefs}
\end{figure}

\begin{figure}[h]
    \centering
    \includegraphics[width=12cm]{uml_sequence_checkout.png} % à remplacer
    \caption{Séquence : ajout au panier et paiement}
\end{figure}

\section{Conclusion de conception}
La modélisation UML a servi de support pour valider les responsabilités, limiter
le couplage et préparer les contrôleurs et vues. Elle a réduit les risques
d'incohérence entre parcours utilisateur et modèle de données.

%========================================================================================
\chapter{Réalisation}

\section{Environnement de développement}
\begin{itemize}
    \item \textbf{Backend} : ASP.NET Core MVC, contrôleurs structurés, routes dédiées (\texttt{Boutique}, \texttt{Cart}, \texttt{AdminProducts}, \texttt{Account}).
    \item \textbf{Données} : EF Core, SQL Server, seeding initial dans \texttt{Program.cs} pour disposer de catégories et parfums dès le premier lancement.
    \item \textbf{Front} : Razor + Bootstrap 5, vues responsives, jQuery pour la validation et quelques interactions.
    \item \textbf{Sécurité} : authentification cookie, rôles, validation antiforgery, vérification des sessions pour le panier.
    \item \textbf{Configuration} : \texttt{appsettings.json}, injection de dépendances, middleware (static files, auth, session).
\end{itemize}

\section{Fonctionnalités principales}
\subsection*{Boutique (publique)}
\begin{itemize}
    \item Liste des parfums avec filtres par catégorie et recherche.
    \item Fiche détail (description, prix, visuel, stock).
    \item Ajout rapide au panier avec gestion des quantités et taille par défaut.
\end{itemize}

\subsection*{Panier et commande}
\begin{itemize}
    \item Stockage panier en session, mise à jour des quantités, suppression, vidage.
    \item Paiement simulé (saisie adresse et carte), récapitulatif masqué des numéros.
    \item Création de la commande et des lignes en base, génération d'un identifiant type \texttt{CMD-000123}.
    \item Historique des commandes pour l'utilisateur connecté ; suivi global pour l'admin.
\end{itemize}

\subsection*{Administration}
\begin{itemize}
    \item Tableau de bord synthétique (nombre de produits, catégories, stock global).
    \item CRUD complet sur les parfums avec sélection de catégorie.
    \item Suivi des commandes et mise à jour du statut (admin uniquement).
\end{itemize}

\subsection*{Authentification et rôles}
\begin{itemize}
    \item Inscription et connexion par email/mot de passe.
    \item Rôle \textit{Admin} donnant accès au back-office et aux vues dédiées.
    \item Sessions persistantes configurées via cookies sécurisés.
\end{itemize}

\section{Extraits notables}
\begin{itemize}
    \item \texttt{Program.cs} : configuration des services (auth cookie, session, EF Core, routes) et \textbf{seeding} des catégories/parfums pour démarrer avec des données.
    \item \texttt{BoutiqueController} : navigation publique, filtrage par catégorie et recherche plein texte.
    \item \texttt{CartController} : gestion du panier en session, paiement simulé, création des commandes et historisation.
    \item \texttt{AdminProductsController} : sécurisation par rôle \textit{Admin}, dashboard et CRUD du catalogue.
\end{itemize}

\section{Conclusion de réalisation}
L'implémentation suit le modèle MVC pour séparer présentation, logique métier et
accès aux données. Les seeds facilitent la démonstration immédiate. La sécurité
par cookies et rôles protège le back-office tandis que les protections CSRF et la
validation côté serveur renforcent la robustesse de l'application.

%========================================================================================
\chapter*{Conclusion générale et perspectives}
\addcontentsline{toc}{chapter}{Conclusion générale et perspectives}

Le projet \textbf{Gestion de Parfums} fournit une base opérationnelle pour une
parfumerie en ligne : catalogue, panier, commandes et administration des stocks.
Les prochaines améliorations possibles sont :
\begin{itemize}
    \item Intégrer une passerelle de paiement réelle (Stripe, PayPal).
    \item Ajouter des alertes de stock et un reporting exportable (PDF/Excel).
    \item Ouvrir une API publique sécurisée pour une application mobile.
    \item Internationaliser (langues, devises) et ajouter des recommandations produit.
\end{itemize}

\end{document}

